\section{Introduction}
\label{sec:intro}

The indices most commonly used to estimate a predator's food preferences, or selectivity, are relatively old \citep{Ivlev:1964,Jacobs:1974,Chesson:1978,Strauss:1979,Vanderploeg:1979,Chesson:1983}, and yet many applied papers continue to use them; a quick search of just $2014$ returns hundreds of publications that cite these fundamental papers, a few being \citet{Clements:2014,Hansen:2014,Hellstrom:2014,Lyngdoh:2014,Madduppa:2014}.  These indices, though intuitive, lack the statistical rigor of a model, focus on a snapshot in time, and rarely allow more than one prey species to be considered \citep{Lechowicz:1982}.  We propose an intuitive statistical model to determine and statistically test differences in a predators' prey preferences across an array of time points and between multiple prey species.   

A comprehensive overview by \citet{Lechowicz:1982}, which was later summarized by \citet{Manly:1992}, details the benefits and faults of the most popular indices.  According to these reviews, a majority of the indices give comparable results, save Strauss' linear index $L$, despite the fact that most of the methods differ by range and linearity of response.  While \citet{Lechowicz:1982} recommends one index, $E^*$ by \citet{Vanderploeg:1979} as the ``single best,'' albeit imperfect, index, \citet{Manly:1992} instead take the approach of suggesting the indices which  estimate biologically meaningful values.  \citet{Lechowicz:1982} recommends the index $E^*$, an element of the \citet{Manly:1992} suggested indices, because the index value $0$ denotes random feeding, the index has a range restricted to $[-1,1]$ (though $E^*=1$ is nigh impossible), and because the index is based on the predator's choice of prey as a function of both the availability of the prey as well as the number of available prey types (assumed known).  The downside to this index is its lack of reasonable statistical properties \citep{Lechowicz:1982}, thus making the computation of standard errors, and hypothesis testing difficult.  This is, in fact, a common fault amongst most of the indices.  

To encourage more formal statistical inference, and simultaneously generalize predators' selectivity to animal resource selection, \citet{Manly:1992} proposed the use of generalized linear models (GLM).  The well established literature on GLMs allows for hypothesis testing to replace the indices, by estimating the proportion of eaten prey species to that which is available while using environmental variables as predictors.  The model we present here, while restricted to predators' preferences, is a compromise between these two extremes, indices and GLMs.  Our model offers formal hypothesis testing and inference similar to the GLMs of \citet{Manly:1992}, but also provides meaningful single number summaries of the predator's dietary preferences.  To do this, we estimate the rate at which a predator consumes the prey of interest instead of estimating the proportion of consumed to available prey.  An outcome of our model, is that we give up the somewhat arbitrary preference for an index to have the range $[-1,1]$, and random feeding, now denoted by $1$, is formally testable across time points and across prey species.

Our model enables formal hypothesis testing and statistical inference, while being general enough to perform statistical tests across multiple species and time points.  This provides researchers a more detailed analysis of the predator's eating preferences.  Further, because our model is based on underlying Poisson distributions, members of the well studied exponential family, we are able to estimate the parameters of interest even when exact tallies of the number of each prey species eaten within any given time period is not observed.  Instead, we rely on the researcher being able to DNA sequence the contents of the predator's gut and make a simple binary conclusion: this predator ate some of that prey species during this time period, or did not.  

This paper is organized as follows.  Section~\ref{sec:methods} describes our statistical model, for both fully observed count data, and for the non-observed count data for which we use the Expectation-Maximization (EM) algorithm, and the statistical tests used to make statements about the population parameters of interest.  In section~\ref{sec:sim}, we offer a simulation study that demonstrates the accuracy of our methods.  Section~\ref{sec:data} provides a real dataset, which investigates the eating preferences of the wolf spider, found in the Berea College Forest in Madison County, Kentucky, USA, to demonstrate how interested practitioners could apply our methods.  A brief discussion concludes the paper in section~\ref{sec:discussion}.  Alongside our model, we offer an \texttt{R} \citep{Core-Team:2014} package named \texttt{spiders} that fits all the methods discussed.  

%%% Local Variables: 
%%% mode: latex
%%% TeX-master: "main"
%%% End: 
