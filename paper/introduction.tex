\section{Introduction}
\label{sec:intro}

Modeling a predator's food preferences, or selectivity, is a relatively old topic \citep{Ivlev:1964,Jacobs:1974,Chesson:1978,Strauss:1979,Vanderploeg:1979,Chesson:1983}, and yet continues to generate numerous publications.  Many of the indices developed, though intuitive, focused on a snapshot in time, few allow more than one prey species, and practical asymptotic statistical properties are even more rare.  This restricts the interested practitioner to the most computationally manageable of techniques, and has completely ignored simultaneous testing across an array of both time and species.

A comprehensive overview by \citet{Lechowicz:1982} details the benefits and faults of the most common methods at that time.  It was demonstrated there, that a majority of the indices give comparable results, save Strauss' linear index $L$, despite the fact that most of the methods differ by range and linearity of response.  Lechowicz generally finds that indices are either biologically interpretable, but statistically intractable, or are of little practical use.  By the end of the paper, the selectivity index $E^*$ by \citet{Vanderploeg:1979} is recommended as the ``single best,'' albeit imperfect, index.  This index is recommended on the basis of random feeding denoted by the index value $0$, a possible range of $[-1,1]$ (though $E^*=1$ is nigh impossible), and because the index is based on the predator's choice of prey as a function of both the availability of the prey as well as the availability of all other prey.  The downside to this index is its lack of reasonable statistical properties.

A few years later, \citet{Chesson:1983} developed a stochastic waiting time model that assumes the searching time necessary to find a prey of interest follows an exponential distribution.  This model, due to the memory-less property of the exponential distribution, inherently assumes that when a plausible prey species is encountered, but not eaten, that the time to find the next prey is again exponentially distributed.  Thus, this model naturally considers the availability of other prey species by means of the predator's search time.  We feel this model while intuitive in theory does not accord well with the data collection of the other indices.  Therefore, we propose, loosely speaking, to invert Chesson's model and quantify the expected number of prey encounters by the predator of interest.  To do this, our model uses the Poisson distribution to measure the rate at which encounters of a given prey species happens and the rate at which a predator eats said prey.  In the process, however, we give up the somewhat arbitrary preference for an index to have $0$ denote random eating and the intuitive range of $[-1,1]$.  

The model presented here maintains tractable asymptotic properties while being general enough to take into account multiple species and time points.  By modeling both time and any number of prey species, we are able to see a more detailed analysis of the predator's eating preferences.  The simplicity of the model allows us to consider predators for which exact tallies of the number of each prey species eaten within any given time period is not observed.  Instead, we rely on the researcher being able to DNA sequence the contents of the predator's gut and make a simple binary conclusion: this predator ate some of that prey species during this time period, or did not.  Despite the data's lost information, under certain situations our model is able to accurately estimate the parameters of interest.

%%% Local Variables: 
%%% mode: latex
%%% TeX-master: "main"
%%% End: 
