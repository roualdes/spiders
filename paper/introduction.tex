\section{Introduction}
\label{sec:intro}

The indices most commonly used to estimate a predator's food preferences, or selectivity, are relatively old \citep{Ivlev:1964,Jacobs:1974,Chesson:1978,Strauss:1979,Vanderploeg:1979,Chesson:1983}, and yet continue to generate numerous publications; a quick search returns hundreds of publications that cite these fundamental papers, many of which are quite recent \citep{Clements:2014,Hansen:2014,Hellstrom:2014,Lyngdoh:2014,Madduppa:2014}.  These indices, though intuitive, lack statisical justification, focus on a snapshot in time, few allow more than one prey species to be considered, and practical asymptotic statistical properties are even more rare \citep{Lechowicz:1982}.  We propose an intiutive statistical model to determine and statistically test predators' prey preferences across an array of time and multiple prey species.   

A comprehensive overview by \citet{Lechowicz:1982}, which was later summarized by \citet{Manly:1992}, details the benefits and faults of the most popular indices.  According to these reviews, a majority of the indices give comparable results, save Strauss' linear index $L$, despite the fact that most of the methods differ by range and linearity of response.  While \citet{Lechowicz:1982} recommends just one index, $E^*$ by \citet{Vanderploeg:1979} as the ``single best,'' albeit imperfect, index, \citet{Manly:1992} instead takes the approach of suggesting against the subset of indices which don't ``estimate any biologically meaningful value.''  $E^*$, an element of the \citet{Manly:1992} suggested indices, is recommended on the basis of random feeding denoted by the index value $0$, a possible range of $[-1,1]$ (though $E^*=1$ is nigh impossible), and because the index is based on the predator's choice of prey as a function of both the availability of the prey as well as the availability of all other prey.  The downside to this index is its lack of reasonable statistical properties \citep{Lechowicz:1982}, thus making standard errors and hypothesis testing somewhat of a mystery.  This is, in fact, a common fault amongst most of the indices.  

To encourage more formal statistical inference, and simultaneously generalize predators' selectivity to animal resource selection, \citet{Manly:1992} proposed the use of generalized linear models (GLM).  The well established literature on GLMs allowed for hypothesis testing to replace the indices.  The model we present here, while restricted to predators' eating preferences, is a comprimise between these two extremes, indices and GLMs.  Our model offers formal hypothesis testing and inference similar to the GLMs of \citet{Manly:1992}, but also provides meaningfull single number summaries of the predator's dietary preferenes.  To do this, we estimate the rate at which a predator consumes the prey of interest instead of estimating the proportion of consumed to available prey.  An outcome of our model, is that we give up the somewhat arbitrary preference for an index to have $0$ denote random eating and the range of $[-1,1]$.  

Our model enables formal hypothesis testing and statistical inference, while being general enough to perform statistical tests across multiple species and time points.  This provides researchers a more detailed analysis of the predator's eating preferences.  Further, because our model is based on the Poisson distribution, members of the well studied exponential family, we are able to estimate the parameters of interest even when exact tallies of the number of each prey species eaten within any given time period is not observed.  Instead, we rely on the researcher being able to DNA sequence the contents of the predator's gut and make a simple binary conclusion: this predator ate some of that prey species during this time period, or did not.  Alongside our model, we offer an \texttt{R} \cite{Core-Team:2014} package named \texttt{spiders} that fits all of our methods.  

%%% Local Variables: 
%%% mode: latex
%%% TeX-master: "main"
%%% End: 
