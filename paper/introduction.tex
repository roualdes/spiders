\section{Introduction}
\label{sec:intro}

Modeling a predator's food selection was given much attention for a short period of time; see \citet{Strauss:1979} and the references within.  Many, if not all, of the indices developed, though intuitive, focused on a snapshot in time and only on one prey species, and they rarely had practicle asymptotic properties.  This has left interested practicioners to the most computationally manageable of the techniques, and has completely ignored simultaneous testing across an array of both species and time.

The model presented here, maintains tractable asymptotic properties while being general enough to take into account an array of speices and time points.  By modeling both time and any number of prey species, we are able to see a more detailed anlyais of the predator's eating preferences.  Further, the simplicity of the model allows us to consider predators for which exact tallies of the number of each variety of prey species eaten within any given time period is not observed.  Instead, we rely on the researcher being able to DNA sequence the contents of the predator's gut, and make a simple binary conclusion: yes this predator ate some of that prey species during this time period, or no they did not.  Under certain situations, our models is able to accurately estimate parameters of interest despite having lost much information.  

To showcase our model, we performed simulations for both scenarios when full count data from the predators' guts are observed, and when instead only binary repsonses, indicating specific prey species were eaten, are observed.  Our \texttt{R} \citep{R:2014} package, named \texttt{spiders}, is available on CRAN so that the broader community of researchers can similarly apply our methods.

%%% Local Variables: 
%%% mode: latex
%%% TeX-master: "main"
%%% End: 
