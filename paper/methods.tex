\section{Methods}
\label{sec:methods}

\subsection{Data}

We assume data are collected in the following manner.  Traps are dispersed, for $T$ time periods, throughout the habitat of the predator and prey of interest.  Prey species, indexed by $s \in \{1, \ldots, S \}$, are collected in the traps and counted at each time period.  The number of prey species $s$ the predator will encounter on average during time period $t \in \{1, \ldots, T\}$ are considered random draws from a Poisson distribution with rate parameter $\gamma_{st}$.  We further assume that the number of prey species found in the gut of the similarly trapped predators follows a Poisson distribution with rate $\lambda_{st}$.  Here, the parameter $\lambda_{st}$ represents the rate at which the predator ate prey species $s$ during time period $t$.  By modeling $\lambda_{st}$ and $\gamma_{st}$ we are able to test claims about a predator's eating preferences.  

The use of Poisson distributions make the following implicit assumptions: $1)$ traps independently catch the prey species of interest, $2)$ predators eat independent of each other.

We denote the number of predators and the number of prey species caught, in each time period $t$, by $J_t$ and $I_t$, respectively.  Let $X_{jst} \iid \mathcal{P}(\lambda_{st})$ represent the number of prey species $s$ that predator $j$ ate during occurrence $t$, where $j \in \{1, \ldots, J_t\}$.  Let $Y_{ist} \iid \mathcal{P}(\gamma_{st})$ represent the number of prey species $s$ found in trap $i$ during occurrence $t$, $i \in \{1, \ldots, I_t\}$.  Formal statistical statements about the relative magnitudes of the parameters $\boldsymbol{\lambda}$ and $\boldsymbol{\gamma}$ offer insights to the relative rates at which predators eat particular prey species.  

We consider five variations on the relative magnitude of $\lambda_{st}/\gamma_{st} = c_{st}$.  These five hypotheses each allow $c_{st}$ to vary by time, prey species, both, or neither.  Because the five hypotheses are nested, a natural testing order is suggested in Figure~\ref{fig:hier}.

\begin{enumerate}
\item $c_{st} = 1$
\item $c_{st} = c$
\item $c_{st} = c_s$
\item $c_{st} = c_t$
\item $c_{st} = c_{st}$
\end{enumerate}

The first hypothesis states that predators and traps sample all prey species at the same rate.  One imagines this is the case if the predator simply eats prey which comes within its reach, thus suggesting no selection for a particular prey item.  The second states that predators sample prey proportionally across all time periods.  The third hypothesis states that predators sample different prey species at different rates, but each rate is steady across time.  This implies that the predator expresses preferences for one prey species over another, but is unresponsive to changes due to time.  Conversely, the forth hypothesis implies that each prey species is sampled similarly within each time period, while the rates across time are allowed to change.  The fifth hypothesis assumes a predator's selection varies by both time and prey species.  This would make sense if environmental and biological variables, such as weather, prey availability, and/or palatability were affecting predators' selection strategies.  

\begin{figure}
  \centering
  \begin{tikzpicture}[->,>=stealth',shorten >=1pt,auto,node distance=2.8cm, semithick]

    \node[state] (A)                    {$1$};
    \node[state] (B) [below of=A]       {$c$};
    \node[state] (C) [below left of=B]  {$c_s$};
    \node[state] (D) [below right of=B] {$c_t$};
    \node[state] (E) [below left of=D]  {$c_{st}$};

    \path       (A) edge node {} (B)
                (B) edge node {} (C)
                    edge node {} (D)
                (C) edge node {} (E)
                (D) edge node {} (E);
  \end{tikzpicture}
  
  \caption{This hierarchy of hypotheses suggests the order in which the discussed models should be tested.  One begins with the models at the top and sequentially, following the arrows, tests the hypotheses using the formal test described in section~\ref{sec:test}.}
  \label{fig:hier}
\end{figure}

\subsection{Fully Observed Count Data}
\label{sec:count}

The likelihood function that allows for estimation of these parameters is as follows.  Since we assume $X_{jst}$ is independent of $Y_{ist}$ we can simply multiply the respective Poisson probability density functions, and then form products over all $s,t$ to obtain the likelihood.  

\begin{equation}
  \label{eq:likelihood}
  L(x_{jst}, y_{ist} |\boldsymbol{\lambda}, \boldsymbol{\gamma}) = \prod_{t = 1}^{T} \prod_{s=1}^S \left\{ \prod_{j=1}^{J_t} f_X(x_{jst}|\boldsymbol{\lambda}) \prod_{i=1}^{I_t} f_Y(y_{ist} | \boldsymbol{\gamma}) \right\}.
\end{equation}

\noindent Writing all five hypotheses as $\lambda_{st} = c_{st}\gamma_{st}$, we can, in the simplest cases, find analytic solutions for the maximum likelihood estimates of $c_{st}$ and $\gamma_{st}$.  Under the hypothesis $c_{st} = 1$, and when the data are balanced $J_t = J$, $I_t = I$, and $c_{st} = c$ analytic solutions exist.  Namely, these solutions are

\begin{equation*}
  \hat{\gamma}_{st} = \frac{X_{\cdot st} + Y_{\cdot st}}{J_t + I_t}, \quad \text{ and } \quad \hat{c} = \frac{I \sum_{s,t} X_{\cdot st}}{J \sum_{s,t} Y_{\cdot st}}, \quad \hat{\gamma}_{st} = \frac{X_{\cdot st} + Y_{\cdot st}}{I \left( \frac{\sum_{st} X_{\cdot st}}{\sum_{st} Y_{\cdot st}} + 1 \right)}
\end{equation*}

\noindent respectively, where $X_{\cdot st} = \sum_{j=1}^{J_t}X_{jst}$ and $Y_{\cdot st} = \sum_{i=1}^{I_t} Y_{ist}$.

In all other cases, analytic solutions are not readily available and instead we rely on the fact that the log-likelihood $l(\boldsymbol{\lambda}, \boldsymbol{\gamma}) = \log{L}$ is concave.  We maximize the log-likelihood, using coordinate descent \citep{Luo:1992}, by iteratively solving partial derivatives of $l$, with respect to $c_{st}$ and $\gamma_{st}$, set equal to zero

\begin{equation*}
  \hat{c} = \frac{\sum_{s,t} X_{\cdot st}}{\sum_t J_t \sum_s \gamma_{st}}, \quad \hat{c}_t =  \frac{\sum_s X_{\cdot st}}{J_t \sum_s \gamma_{st}}, \text{ or} \quad \hat{c}_s = \frac{\sum_{t}X_{\cdot st}}{\sum_t J_t \gamma_{st}}, \text{ and } \quad \hat{\gamma}_{st} = \frac{X_{\cdot st} + Y_{\cdot st}}{J_t c_{st} + I_t}.
\end{equation*}

\subsection{Unobserved Counts}
\label{sec:noncount}

In many applications, such as DNA-based gut-content analysis, it is not possible to count the number of individuals of each prey species that are in a predator's gut.  Instead, it is only possible to detect whether or not a predator consumed the prey species during a given time period, based on the rate at which prey DNA decays in the predator gut \citep{Greenstone:2013}.  In this case we can still make inference about the predators' preferences for the different prey species by using the Expectation-Maximization (EM) algorithm to compute maximum likelihood estimates. 

We denote the binary random variable indicating that the $j^{th}$ predator did in fact eat at least one individual of prey species $s$ in time period $t$ by $Z_{jst} = 1(X_{jst} > 0)$.  The variables are independent Bernoulli observations with success probability $p_{st} = P(Z_{jst}=1)= 1-\exp\{-\lambda_{st}\}$.  Despite not observing $X_{jst}$, we can compute maximum likelihood estimates of the parameters $\boldsymbol{\lambda}, \boldsymbol{\gamma}$ through the EM algorithm using the complete data log-likelihood
\[
l_{comp}(\boldsymbol{\lambda}, \boldsymbol{\gamma}) = \log f_{X,Y,Z}(\boldsymbol x, \boldsymbol y, \boldsymbol z|\boldsymbol{\lambda}, \boldsymbol{\gamma}) = \sum_{s=1}^{S} \sum_{t=1}^T \left[ \sum_{j=1}^{J_t} \log f_{X,Z}(x_{jst},z_{jst}|\boldsymbol{\lambda}) + \sum_{i=1}^{I_t}\log f_Y(y_{jst}|\boldsymbol{\gamma}) \right].
\]

The density of $Y_{jst}$ is exactly as in section~\ref{sec:count} and so we focus on deriving the joint density of $X_{jst}$ and $Z_{jst}$.  With the distribution of $Z_{jst}$ given above, we can compute $f_{X,Z}(x_{jst},z_{jst}|\boldsymbol{\lambda})$ by noting that $X_{jst}=0$ with probability 1 if $Z_{jst}=0$, and that $[X_{jst}|Z_{jst}=0]$ has a truncated Poisson distribution with density
\[
  f_{X|Y,Z,\boldsymbol{\lambda},\boldsymbol{\gamma}}(x_{jst}|z_{jst}) =
  \frac{\exp{\{-\lambda_{st}\}} \lambda_{st}^{x_{jst}}}{(1 - \exp{\{-\lambda_{st}\}}) x_{jst}!}1(x_{jst} > 0)
\]
and expected value
\[
\E_{X|Y,Z}X_{jst} = \frac{\lambda_{st} \exp{\{\lambda_{st} \}}}{\exp{\{ \lambda_{st} \}} - 1}.
\]
\noindent The joint density of $X_{jst}, Z_{jst}$ is then 
\begin{equation*}
    f_{X,Z|\boldsymbol{\lambda}}(x_{jst},z_{jst}) = \left\{
    \begin{array}{lr}
      \exp{\{ -\lambda_{st} \}}, & x_{jst}=0 \mbox{ and } z_{jst} = 0 \\
      \frac{\exp{\{-\lambda_{st} \}} \lambda_{st}^{x_{jst}}}{x_{jst}!}, & x_{jst} > 0 \mbox{ and } z_{jst} = 1 \\
      0 & \mbox{otherwise}
    \end{array}
  \right..
\end{equation*}

The EM algorithm works by iterating two steps, the E-step and M-step, until the optimum is reached \citep{Dempster:1977,McLachlan:2007}.  Let $k$ index the iterations in the EM algorithm so that $\boldsymbol \lambda^{(k)}$ and $\boldsymbol \gamma^{(k)}$ denote the estimates computed on the $k^{th}$ M-step. The E-step consists of computing the expectation of $l_{comp}$ with respect to the conditional distribution of $X$ given the current estimates of the parameters
\[
Q^{(k)}(\boldsymbol{\lambda},\boldsymbol{\gamma}) = \E_{X|Y,Z,\boldsymbol{\lambda}^{(k)}} l_{comp}
\]
in order to remove the unobserved data. The M-step then involves maximizing $Q = \E l_{comp}$ with respect to the parameters in the model to obtain updated estimates of the parameters,
\[
(\boldsymbol{\lambda}^{(k+1)}, \boldsymbol{\gamma}^{(k+1)}) = \argmax_{(\boldsymbol{\lambda}, \boldsymbol{\gamma})} Q^{(k)}(\boldsymbol{\lambda}, \boldsymbol{\gamma}).
\]
These steps are then alternated until a convergence criterion monitoring subsequent differences in the parameter estimates/likelihood is met.  

The calculation of $Q^{(k)}(\boldsymbol{\lambda}, \boldsymbol{\gamma})$ is not difficult and is given by:
\begin{align}
  \label{eq:estep}
  \begin{split}
  Q^{(k)}(\boldsymbol{\lambda}, \boldsymbol{\gamma})
  & = \E \log{f_{X,Z|\boldsymbol{\lambda}}(X_{jst},z_{jst})} + \log{f_{Y|\boldsymbol{\gamma}}(y_{ist})} \\
  & = \sum_{s=1}^S \sum_{t=1}^T \sum_{j=1}^{J_t} \E \log{f_{X,Z|\boldsymbol{\lambda}}(X_{jst},z_{jst})}
  + \sum_{s=1}^S \sum_{t=1}^T \sum_{i=1}^{I_t} \log{f_{Y|\boldsymbol{\gamma}}(y)} \\
  & \propto \sum_{s,t,j} \left( - \lambda_{st} 
    + z_{jst} \log{\lambda_{st}} \E X_{jst} \right) + \sum_{s,t} \left( -I_t \gamma_{st} + Y_{\cdot st} \log{I_t \gamma_{st}} \right) \\
  & \propto \sum_{s,t} \left( -J_t \lambda_{st} + z_{\cdot st} \log{\lambda_{st}} \E (X_{jst}|\lambda_{st}^{(k)},\gamma_{st}^{(k)}) \right) + \sum_{s,t} \left( -I_t \gamma_{st} + Y_{\cdot st} \log{I_t \gamma_{st}} \right).
\end{split}
\end{align}
\noindent No analytic solution to the M-step exists, however, so we chose to maximize $Q$ with coordinate descent \citep{Luo:1992}.  In fact, as we only need to find parameters that increase the value of $Q$ on each iteration, we forgo fully iterating to find the maximum and instead perform just one step uphill within each EM iteration.  Since $Q^{(k)}$ is concave and smooth in the parameters $\boldsymbol{\lambda}, \boldsymbol{\gamma}$, we are able to use the convergence of parameter estimates, $||(\boldsymbol{\lambda}^{(k)}, \boldsymbol{\gamma}^{(k)}) - (\boldsymbol{\lambda}^{(k+1)}, \boldsymbol{\gamma}^{(k+1)})||_{\infty} < \tau $, for some $\tau>0$, as our stopping criterion.

As we show in our simulation study, this generalized EM algorithm accurately estimates the parameters when values of $\lambda_{st}$ are relatively small, such that zeros are prevalent in the data $Z_{jst}$.  In this case, not too much information is lost since estimation of $\E Z_{jst}$ can be estimated well by the proportion of observed zeros.  In contrast, if the predator consistently eats a given prey species, few to no zeros will show up in the observed data and $\E Z_{jst}$ is estimated to be nearly $1$.  The loss of information is best seen by attempting to solve for $\lambda_{st}$ in the equation $1 = \E Z_{jst} = 1 - \exp\{-\lambda_{st}\}$.  As the proportion of ones in the observed data increases, we expect $\lambda_{st}$ to grow exponentially large.  When no zeros are present in the data, so that where only ones are observed, the likelihood can be made arbitrarily large by sending the parameter off to infinity.  

\subsection{Testing}
\label{sec:test}
The likelihood ratio test statistic is

\begin{equation*}
  \label{eq:LRT}
    \Lambda(X,Y) := -2 \log{ \frac{ \sup_{\theta_0} L(\theta_0|X,Y)}{ \sup_{\theta_1} L(\theta_1|X,Y)} },
\end{equation*}

\noindent where $\theta_0, \theta_1$ represent the parameters estimated under the null and alternative hypotheses, respectively.  It is well known that the asymptotic distribution of $\Lambda$ is a $\chi_{\rho}^2$ distribution with $\rho$ degrees of freedom \citep{Wilks:1938}.  When the observations $X_{jst}$ are not observed, we use $L_{obs}(Z,Y)$ as the likelihood in the calculation of $\Lambda$.  

The degrees of freedom $\rho$ equal the number of free parameters available in the stated hypotheses under question.  If we put the null hypothesis to be $H_0: \lambda_t = c_t \gamma_t$, for all $t$ and contrast this against $H_1: \lambda_{st} = c_{st}\gamma_{st}$ then there are $\rho = 2(S \cdot T) - S \cdot T - T = S \cdot T - T$ degrees of freedom.

A set of hypotheses is determined by the p-value of the $\chi^2_{\rho}$ distribution.  Hence, with a level of significance, $\alpha$, the null hypothesis is rejected in favor of the alternative hypothesis if $\mathbb{P}(\chi^2_{\rho} > \Lambda) < \alpha$.  

\subsection{Linear Transformations of $c_{st}$}

After determining which model best fits the data, more detail can be extracted through a hypothesis test of the elements of $c_{st}$, or in vector notation as $\mathbf{c} \in \mathbb{R}^{S\cdot T}$.  Let the elements of $\hat{\mathbf{c}}$ be the maximum likelihood estimates, $\hat{c}_{st}$, as found via the framework above.  Since $\hat{\mathbf{c}}$ is asymptotically normally distributed, any linear combination of the elements is also asymptotically normally distributed.  For instance, let $a$ be a vector of the same dimension of $\hat{\mathbf{c}}$.  Then $a^t\hat{\mathbf{c}}$ is asymptotically distributed as $\mathcal{N}(a^t\mathbf{c}, a^t\Sigma a)$, where $\Sigma$ is the covariance matrix of the asymptotic distribution of $\hat{\mathbf{c}}$.  

Suppose, for example, that the hypothesis $c_s$ is determined to best fit the data with $s$ ranging $s = 1, 2, 3$.  We can test to see whether or not two species are statistically equally preferred under the null hypothesis $c_{1} = c_{2}$.  This hypothesis is alternatively written in vector notation as $a^t\mathbf{c} = 0$, where $a = (1, -1, 0)^t$.  Tests of the following form $H_0: a^t\mathbf{c} = \mu$ against any alternative of interest are then approximate $Z$-tests.  Confidence intervals of any size are similarly, readily obtained.  Of course, one could also use a $t$ distribution as a small sample size correction.

%%% Local Variables: 
%%% mode: latex
%%% TeX-master: "main"
%%% End: 
